%%
%% This is file `sample-acmsmall-submission.tex',
%% generated with the docstrip utility.
%%
%% The original source files were:
%%
%% samples.dtx  (with options: `all,journal,bibtex,acmsmall-submission')
%% 
%% IMPORTANT NOTICE:
%% 
%% For the copyright see the source file.
%% 
%% Any modified versions of this file must be renamed
%% with new filenames distinct from sample-acmsmall-submission.tex.
%% 
%% For distribution of the original source see the terms
%% for copying and modification in the file samples.dtx.
%% 
%% This generated file may be distributed as long as the
%% original source files, as listed above, are part of the
%% same distribution. (The sources need not necessarily be
%% in the same archive or directory.)
%%
%%
%% Commands for TeXCount
%TC:macro \cite [option:text,text]
%TC:macro \citep [option:text,text]
%TC:macro \citet [option:text,text]
%TC:envir table 0 1
%TC:envir table* 0 1
%TC:envir tabular [ignore] word
%TC:envir displaymath 0 word
%TC:envir math 0 word
%TC:envir comment 0 0
%%
%% The first command in your LaTeX source must be the \documentclass
%% command.
%%
%% For submission and review of your manuscript please change the
%% command to \documentclass[manuscript, screen, review]{acmart}.
%%
%% When submitting camera ready or to TAPS, please change the command
%% to \documentclass[sigconf]{acmart} or whichever template is required
%% for your publication.
%%
%%
\documentclass[acmsmall,screen,review]{acmart}
%%
%% \BibTeX command to typeset BibTeX logo in the docs
\AtBeginDocument{%
  \providecommand\BibTeX{{%
    Bib\TeX}}}

%% Rights management information.  This information is sent to you
%% when you complete the rights form.  These commands have SAMPLE
%% values in them; it is your responsibility as an author to replace
%% the commands and values with those provided to you when you
%% complete the rights form.
\setcopyright{acmlicensed}
\copyrightyear{2025}
\acmYear{2025}
\acmDOI{XXXXXXX.XXXXXXX}

%%
%% These commands are for a JOURNAL article.
\acmJournal{JACM}
%\acmVolume{37}
%\acmNumber{4}
%\acmArticle{111}
\acmMonth{7}

%%
%% Submission ID.
%% Use this when submitting an article to a sponsored event. You'll
%% receive a unique submission ID from the organizers
%% of the event, and this ID should be used as the parameter to this command.
%%\acmSubmissionID{123-A56-BU3}

%%
%% For managing citations, it is recommended to use bibliography
%% files in BibTeX format.
%%
%% You can then either use BibTeX with the ACM-Reference-Format style,
%% or BibLaTeX with the acmnumeric or acmauthoryear sytles, that include
%% support for advanced citation of software artefact from the
%% biblatex-software package, also separately available on CTAN.
%%
%% Look at the sample-*-biblatex.tex files for templates showcasing
%% the biblatex styles.
%%

%%
%% The majority of ACM publications use numbered citations and
%% references.  The command \citestyle{authoryear} switches to the
%% "author year" style.
%%
%% If you are preparing content for an event
%% sponsored by ACM SIGGRAPH, you must use the "author year" style of
%% citations and references.
%% Uncommenting
%% the next command will enable that style.
%%\citestyle{acmauthoryear}
\usepackage{spverbatim}

%%
%% end of the preamble, start of the body of the document source.
\begin{document}

%%
%% The "title" command has an optional parameter,
%% allowing the author to define a "short title" to be used in page headers.
\title{TeXorcist}

%%
%% The "author" command and its associated commands are used to define
%% the authors and their affiliations.
%% Of note is the shared affiliation of the first two authors, and the
%% "authornote" and "authornotemark" commands
%% used to denote shared contribution to the research.
\author{Jingren Wang}
%\authornote{Both authors contributed equally to this manual.}
\email{jingrenwangcyber@gmail.com}
%\orcid{1234-5678-9012}
\affiliation{%
  \institution{Hong Kong University of Science and Technology(Guangzhou)}
  \city{Guangzhou}
  \state{Guangdong}
  \country{China}
}


%%
%% By default, the full list of authors will be used in the page
%% headers. Often, this list is too long, and will overlap
%% other information printed in the page headers. This command allows
%% the author to define a more concise list
%% of authors' names for this purpose.
%\renewcommand{\shortauthors}{Trovato et al.}

%%
%% The abstract is a short summary of the work to be presented in the
%% article.
\begin{abstract}
    Inspired by the game \textit{The Textorcist: The Story of Ray Bibbia}, TeXorcist is a compilation of common writing style issues identified through contributor experience. This manual serves as a comprehensive pre-submission checklist to improve document quality before peer review. The guide covers various aspects including (but not limited to) stylistic conventions and common writing fixes - addressing these meticulous details will enhance both formatting consistency and review efficiency.
\end{abstract}

%%
%% The code below is generated by the tool at http://dl.acm.org/ccs.cfm.
%% Please copy and paste the code instead of the example below.
%%
\begin{CCSXML}
<ccs2012>
 <concept>
  <concept_id>00000000.0000000.0000000</concept_id>
  <concept_desc>Do Not Use This Code, Generate the Correct Terms for Your Paper</concept_desc>
  <concept_significance>500</concept_significance>
 </concept>
 <concept>
  <concept_id>00000000.00000000.00000000</concept_id>
  <concept_desc>Do Not Use This Code, Generate the Correct Terms for Your Paper</concept_desc>
  <concept_significance>300</concept_significance>
 </concept>
 <concept>
  <concept_id>00000000.00000000.00000000</concept_id>
  <concept_desc>Do Not Use This Code, Generate the Correct Terms for Your Paper</concept_desc>
  <concept_significance>100</concept_significance>
 </concept>
 <concept>
  <concept_id>00000000.00000000.00000000</concept_id>
  <concept_desc>Do Not Use This Code, Generate the Correct Terms for Your Paper</concept_desc>
  <concept_significance>100</concept_significance>
 </concept>
</ccs2012>
\end{CCSXML}

%\ccsdesc[500]{Do Not Use This Code~Generate the Correct Terms for Your Paper}
%\ccsdesc[300]{Do Not Use This Code~Generate the Correct Terms for Your Paper}
%\ccsdesc{Do Not Use This Code~Generate the Correct Terms for Your Paper}
%\ccsdesc[100]{Do Not Use This Code~Generate the Correct Terms for Your Paper}

%%
%% Keywords. The author(s) should pick words that accurately describe
%% the work being presented. Separate the keywords with commas.
%\keywords{Do, Not, Us, This, Code, Put, the, Correct, Terms, for,
%  Your, Paper}

%\received{20 February 2007}
%\received[revised]{12 March 2009}
%\received[accepted]{5 June 2009}

%%
%% This command processes the author and affiliation and title
%% information and builds the first part of the formatted document.
\maketitle

\section{Rephrasing is all you need}
The peer review process inherently involves substantial effort from all participants. Given the reciprocal nature of scholarly evaluation - where researchers both receive and provide feedback - addressing common technical and stylistic issues prior to submission represents both a practical necessity and professional courtesy. Systematically compiling and incorporating past review feedback serves two critical functions:
\begin{enumerate}
    \item it reduces the burden on reviewers by eliminating easily preventable issues
    \item it establishes a knowledge base for improving future manuscript preparation.
\end{enumerate}

\section{Related paragraph}
All that needs to be pre-checked are listed as bullet points for easier reference.
\subsection{General}
\begin{itemize}
    \item Take advantages of setting new commands such as
\begin{spverbatim}
\newcommand{\Issue}[1]{\textcolor{red}{[\textbf{Discussion needed}: #1]}}
\newcommand{\TODO}[1]{\textcolor{blue}{[\textbf{TODO @Jingren}: #1]}}
\end{spverbatim}
These commands simplifies writing and reviewing process.
    \item Use sub-texfiles instead of merging all contents in one file. One section one sub-texfile.
    \item Always end a sentence with a period. Be careful with misusing of comma everywhere.
    \item Check the use of quotes, use
    \begin{verbatim}
        `` and ''
    \end{verbatim}
    \vspace{-\baselineskip}
    before and after the quotes.
    \item Check the use of ``a''/``an'' before a word.
    \item Use LLM such as GPT/Deepseek to help with rephrasing. Give detailed and concrete prompts.
    \item For different types of dash lines, we have
    \begin{verbatim}
        Hyphen -
        En-dash --
        Em-dash ---
    \end{verbatim}
    \vspace{-\baselineskip}
    Detailed answer about dash usage could be seen \href{https://tex.stackexchange.com/questions/53413/what-is-the-latex-command-for-em-dash}{here} and \href{https://www.reddit.com/r/writing/comments/vysrhj/comment/ig41u4x/?utm_source=share&utm_medium=web3x&utm_name=web3xcss&utm_term=1&utm_content=share_button}{here}.

\end{itemize}
\subsection{Introduction}
\begin{itemize}
    \item Gradually introduce instead of directly give what the paper is doing.
    \item Using one small example to illustrate the idea if possible.
\end{itemize}

\subsection{Background}
\begin{itemize}
    \item Give/Mention definition of each basic term you used during the paper, do not miss any one of them.
\end{itemize}

\subsection{Table \& Figure}
\begin{itemize}
    \item Giving exact minimized example is better than giving an abstract and vague graph.
    \item Color better results such as using method
    \begin{verbatim}
        \cellcolor{green!25}\textbf{236}
    \end{verbatim}
    \vspace{-\baselineskip}
    is more expressive than labeling arrows in the table.
    \item Give label to all the different objects.
    \item Make as detailed description as possible in \verb+\caption+ rather than in the paragraph, reviewers may take time look into examples rather than paragraph.
\end{itemize}

\subsubsection{Rendering}
\begin{itemize}
    \item Try to make the table following the section order, normally we use the parameter \verb+h+ which indicates ``here'', in \verb+\begin{figure}[h]+.
    \item It's not recommended at the first place, but you could modify the space before and after the table by \verb+\vspace{-2em}+.
    
\end{itemize}

\subsection{Equations, variables, functions}
\begin{itemize}
    \item For variables or function names with multiple letters, use \verb+\mathit{}+ to formalize, a counter example could be given:\\
    Compare the variable such as $nMFFCVol$ and $\mathit{nMFFCVol}$, letters are closer in the latter format.

\end{itemize}

\subsection{References}
\begin{itemize}
    \item Add \textasciitilde{ } before each one of the citation/reference which created a space before targets. Such as
    \begin{verbatim}
        ~\ref{tab:SaturateResults}
    \end{verbatim}
    \vspace{-\baselineskip}
    to reference a table or
    \begin{verbatim}
        ~\cite{FFLC}
    \end{verbatim}
    \vspace{-\baselineskip}
    to reference a paper.
    \item Unlike the \verb+\cite{}+, no \verb+~+ is needed when adding \verb+\footnote{}+.
\end{itemize}
\end{document}
\endinput
%%
%% End of file `sample-acmsmall-submission.tex'.
